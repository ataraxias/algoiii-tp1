\documentclass[11pt, a4paper,spanish]{article}
\usepackage[spanish,activeacute]{babel}
\usepackage[utf8]{inputenc}
\usepackage{caratula}
\usepackage{graphicx}
\usepackage{fancyhdr}
\usepackage{amsfonts}
\usepackage{amssymb}
\usepackage{amsmath}
\usepackage{hyperref}
\usepackage{float}	

%\usepackage[ruled,vlined,boxed,commentsnumbered]{algorithm2e}
	
% Encabezado y Pie de pagina
\pagestyle{fancy}
\fancyhf{}
\fancyfoot[L]{\leftmark} 
\fancyhead[L]{Algoritmos y Estucturas de Datos III}
\fancyhead[R]{Trabajo Pr\'actico 1}
\fancyfoot[R]{\thepage}


\begin{document}

% ------------ CARATULA ------------
    
   \materia{Algoritmos y Estructuras de Datos III}
    \submateria{Trabajo Pr\'actico 1}

 

    \titulo{}

	\subtitulo{}

    \integrante{Agust\'in Cangiani}{344/09}{cangiani@gmail.com}

    \integrante{Marco Vanotti}{}{marcovanotti15@gmail.com}

    \integrante{Romina G\'omez}{590/08}{rominagomez1789@gmail.com}

    \integrante{Ver\'onica Coy}{652/08}{verocoy@gmail.com}

    \director{Ceros de funciones, Newton, Bisecci\'on, Energ\'ia mec\'anica, tolerancia, presición}{}

	\maketitle

	\pagebreak

	



% ------------ INDICE ------------
    
    \thispagestyle{empty}
    \tableofcontents
    \pagebreak

% ------------ EJERCICIO 1 ------------

	% ######################################################
% #################### EJERCICIO 1 #####################
% ######################################################

	\section{Ejercicio 1}
% ------------ Introducción ------------
    \subsection{Introducción}

En el primer ejercicio se nos pide dar una respuesta óptima para un problema de máquinas productoras de queso. 

% ------------ DESARROLLO ------------
    \subsection{Desarrollo}

    \pagebreak

% ------------ CORRECTITUD ------------
    \subsection{Correctitud y complejidad}
	% ----------- DEMO -----------
\begin{subsubsection}{Demo}

Vamos a demostrar que si tenemos una secuencia de máquinas óptima, que no esté
ordenada de mayor a menor según el tiempo de producción, podemos cambiar
el orden de dos máquinas adyacentes que estén desordenadas y seguir teniendo
una solución óptima.

Una secuencia de máquinas óptima es una secuencia tal que el Tiempo
de producción de la misma es menor o igual que el de cualquiera de sus
permutaciones.

Definimos una función T para obtener el tiempo total de procesamiento de
una máquina en la posición k de la secuencia S como

\begin{center}
$T(S,k)$ = $\left( \displaystyle\sum\limits_{i=0}^k Costo(S_i) \right)$  + $Peso(S_k)$\\
\end{center}


Definimos la función para calcular el tiempo total de procesamiento de
una secuencia de máquinas como
\begin{center}
$Tiempo(S)$ = max \{ $T(S,j)$, $j$ $\in$ [0 , $\vert S \vert$ )\}
\end{center}


Supongamos que tenemos una secuencia $S$ óptima que no se encuentra ordenada.
Entonces existe un $j$ $<$ $\vert S - 1\vert$ / $Peso(S_j)$ $<$ $Peso(S_{j+1})$.\\


Sea $S'$ la secuencia que se forma de intercambiar la máquina de la posición $j$ con la de la posición $j + 1$, quiero ver que $Tiempo(S)$ $\geq$ $Tiempo(S')$.\\

Sea $i \neq j \wedge i \neq j + 1$, es trivial ver que $T(S,i)$ = $T(S',i)$. Es decir, que si el máximo no era ni $j$, ni $j + 1$, el tiempo total se mantiene igual.\\

Supongamos entonces que $Tiempo(S)$ depende de $T(S, j)$ y $T(S, j + 1)$:\\

\begin{center}

$Tiempo(S)$ = max \{ $T(S,j)$ , $T(S,j+1)$ \}

$T(S,j)$ = $\left(\displaystyle\sum\limits_{i=0}^{j-1} Costo(S_i) \right)$ + $Peso(S_j)$ + $Costo(S_j)$

$T(S,j+1) = \left(\displaystyle\sum\limits_{i=0}^{j-1} Costo(S_i) \right) + Peso(S_{j+1}) + Costo(S_{j+1}) + Costo(S_j)$

Por hipótesis, vale que $Peso(S_j) < Peso(S_{j+1})$

$\Downarrow$


$Tiempo(S)$ = $T(S,j+1)$
\end{center}



\begin{center}
$Tiempo(S')$ = max \{ $T(S',j)$ , $T(S',j+1)$ \}


Como en $S'$ intercambiamos la máquina en la posición $j$ por la máquina en la posición $j + 1$, vale que $S'_j = S_{j+1} \wedge S'_{j + 1} = S_j$

Luego, reemplazando en T las máquinas

$T(S',j)$ = $\left(\displaystyle\sum\limits_{i=0}^{j-1} Costo(S'_i) \right)$ + $Peso(S_{j+1})$ + $Costo(S_{j+1})$

$T(S',j+1)$ = $\left(\displaystyle\sum\limits_{i=0}^{j-1} Costo(S'_i) \right)$ + $Peso(S_j)$ + $Costo(S_j)$ + $Costo(S_{j+1})$

\end{center}

\begin{center}

$Peso(S_{j+1})$ + $Costo(S_{j+1})$ + $Costo(S_j)$ $>$ $Peso(S_j)$ + $Costo(S_j)$ + $Costo(S_{j+1})$\\
$\Downarrow $

$Tiempo(S)$ $\geq$ $Tiempo(S')$\\

\end{center}

\end{subsubsection}


    \pagebreak


% ------------ RESULTADOS ------------
    \subsection{Resultados}
	
	\subsection*{Tests}
	\subsubsection{Gráficos}
    \pagebreak


% ------------ EJERCICIO 2 ------------

	% ######################################################
% #################### EJERCICIO 2 #####################
% ######################################################

	\section{Ejercicio 2}

% ------------ DESARROLLO ------------
    \subsection{Desarrollo}

    \pagebreak

% ------------ CORRECTITUD ------------
    \subsection{Correctitud}
	% ----------- DEMO -----------
\begin{subsubsection}{Demo}

Vamos a demostrar que si tenemos una secuencia de máquinas óptima, que no esté
ordenada de mayor a menor según el tiempo de producción, podemos cambiar
el orden de dos máquinas adyacentes que estén desordenadas y seguir teniendo
una solución óptima.

Una secuencia de máquinas óptima es una secuencia tal que el Tiempo
de producción de la misma es menor o igual que el de cualquiera de sus
permutaciones.

Definimos una función T para obtener el tiempo total de procesamiento de
una máquina en la posición k de la secuencia S como

\begin{center}
$T(S,k)$ = $\left( \displaystyle\sum\limits_{i=0}^k Costo(S_i) \right)$  + $Peso(S_k)$\\
\end{center}


Definimos la función para calcular el tiempo total de procesamiento de
una secuencia de máquinas como
\begin{center}
$Tiempo(S)$ = max \{ $T(S,j)$, $j$ $\in$ [0 , $\vert S \vert$ )\}
\end{center}


Supongamos que tenemos una secuencia $S$ óptima que no se encuentra ordenada.
Entonces existe un $j$ $<$ $\vert S - 1\vert$ / $Peso(S_j)$ $<$ $Peso(S_{j+1})$.\\


Sea $S'$ la secuencia que se forma de intercambiar la máquina de la posición $j$ con la de la posición $j + 1$, quiero ver que $Tiempo(S)$ $\geq$ $Tiempo(S')$.\\

Sea $i \neq j \wedge i \neq j + 1$, es trivial ver que $T(S,i)$ = $T(S',i)$. Es decir, que si el máximo no era ni $j$, ni $j + 1$, el tiempo total se mantiene igual.\\

Supongamos entonces que $Tiempo(S)$ depende de $T(S, j)$ y $T(S, j + 1)$:\\

\begin{center}

$Tiempo(S)$ = max \{ $T(S,j)$ , $T(S,j+1)$ \}

$T(S,j)$ = $\left(\displaystyle\sum\limits_{i=0}^{j-1} Costo(S_i) \right)$ + $Peso(S_j)$ + $Costo(S_j)$

$T(S,j+1) = \left(\displaystyle\sum\limits_{i=0}^{j-1} Costo(S_i) \right) + Peso(S_{j+1}) + Costo(S_{j+1}) + Costo(S_j)$

Por hipótesis, vale que $Peso(S_j) < Peso(S_{j+1})$

$\Downarrow$


$Tiempo(S)$ = $T(S,j+1)$
\end{center}



\begin{center}
$Tiempo(S')$ = max \{ $T(S',j)$ , $T(S',j+1)$ \}


Como en $S'$ intercambiamos la máquina en la posición $j$ por la máquina en la posición $j + 1$, vale que $S'_j = S_{j+1} \wedge S'_{j + 1} = S_j$

Luego, reemplazando en T las máquinas

$T(S',j)$ = $\left(\displaystyle\sum\limits_{i=0}^{j-1} Costo(S'_i) \right)$ + $Peso(S_{j+1})$ + $Costo(S_{j+1})$

$T(S',j+1)$ = $\left(\displaystyle\sum\limits_{i=0}^{j-1} Costo(S'_i) \right)$ + $Peso(S_j)$ + $Costo(S_j)$ + $Costo(S_{j+1})$

\end{center}

\begin{center}

$Peso(S_{j+1})$ + $Costo(S_{j+1})$ + $Costo(S_j)$ $>$ $Peso(S_j)$ + $Costo(S_j)$ + $Costo(S_{j+1})$\\
$\Downarrow $

$Tiempo(S)$ $\geq$ $Tiempo(S')$\\

\end{center}

\end{subsubsection}


    \pagebreak


% ------------ RESULTADOS ------------
    \subsection{Resultados}

    \pagebreak

	
% ------------ EJERCICIO 3 ------------

	 
% ######################################################
% #################### EJERCICIO 3 #####################
% ######################################################

	\section{Ejercicio 3}

% ------------ DESARROLLO ------------
    \subsection{Desarrollo}

    \pagebreak

% ------------ CORRECTITUD ------------
    \subsection{Correctitud}
	% ----------- DEMO -----------
\begin{subsubsection}{Demo}

Vamos a demostrar que si tenemos una secuencia de máquinas óptima, que no esté
ordenada de mayor a menor según el tiempo de producción, podemos cambiar
el orden de dos máquinas adyacentes que estén desordenadas y seguir teniendo
una solución óptima.

Una secuencia de máquinas óptima es una secuencia tal que el Tiempo
de producción de la misma es menor o igual que el de cualquiera de sus
permutaciones.

Definimos una función T para obtener el tiempo total de procesamiento de
una máquina en la posición k de la secuencia S como

\begin{center}
$T(S,k)$ = $\left( \displaystyle\sum\limits_{i=0}^k Costo(S_i) \right)$  + $Peso(S_k)$\\
\end{center}


Definimos la función para calcular el tiempo total de procesamiento de
una secuencia de máquinas como
\begin{center}
$Tiempo(S)$ = max \{ $T(S,j)$, $j$ $\in$ [0 , $\vert S \vert$ )\}
\end{center}


Supongamos que tenemos una secuencia $S$ óptima que no se encuentra ordenada.
Entonces existe un $j$ $<$ $\vert S - 1\vert$ / $Peso(S_j)$ $<$ $Peso(S_{j+1})$.\\


Sea $S'$ la secuencia que se forma de intercambiar la máquina de la posición $j$ con la de la posición $j + 1$, quiero ver que $Tiempo(S)$ $\geq$ $Tiempo(S')$.\\

Sea $i \neq j \wedge i \neq j + 1$, es trivial ver que $T(S,i)$ = $T(S',i)$. Es decir, que si el máximo no era ni $j$, ni $j + 1$, el tiempo total se mantiene igual.\\

Supongamos entonces que $Tiempo(S)$ depende de $T(S, j)$ y $T(S, j + 1)$:\\

\begin{center}

$Tiempo(S)$ = max \{ $T(S,j)$ , $T(S,j+1)$ \}

$T(S,j)$ = $\left(\displaystyle\sum\limits_{i=0}^{j-1} Costo(S_i) \right)$ + $Peso(S_j)$ + $Costo(S_j)$

$T(S,j+1) = \left(\displaystyle\sum\limits_{i=0}^{j-1} Costo(S_i) \right) + Peso(S_{j+1}) + Costo(S_{j+1}) + Costo(S_j)$

Por hipótesis, vale que $Peso(S_j) < Peso(S_{j+1})$

$\Downarrow$


$Tiempo(S)$ = $T(S,j+1)$
\end{center}



\begin{center}
$Tiempo(S')$ = max \{ $T(S',j)$ , $T(S',j+1)$ \}


Como en $S'$ intercambiamos la máquina en la posición $j$ por la máquina en la posición $j + 1$, vale que $S'_j = S_{j+1} \wedge S'_{j + 1} = S_j$

Luego, reemplazando en T las máquinas

$T(S',j)$ = $\left(\displaystyle\sum\limits_{i=0}^{j-1} Costo(S'_i) \right)$ + $Peso(S_{j+1})$ + $Costo(S_{j+1})$

$T(S',j+1)$ = $\left(\displaystyle\sum\limits_{i=0}^{j-1} Costo(S'_i) \right)$ + $Peso(S_j)$ + $Costo(S_j)$ + $Costo(S_{j+1})$

\end{center}

\begin{center}

$Peso(S_{j+1})$ + $Costo(S_{j+1})$ + $Costo(S_j)$ $>$ $Peso(S_j)$ + $Costo(S_j)$ + $Costo(S_{j+1})$\\
$\Downarrow $

$Tiempo(S)$ $\geq$ $Tiempo(S')$\\

\end{center}

\end{subsubsection}


    \pagebreak


% ------------ RESULTADOS ------------
    \subsection{Resultados}

    \pagebreak

	
% ------------ EJERCICIO 4 ------------

	% ######################################################
% #################### EJERCICIO 4 #####################
% ######################################################

	\section{Ejercicio 4}

% ------------ DESARROLLO ------------
    \subsection{Desarrollo}

    \pagebreak

% ------------ CORRECTITUD ------------
    \subsection{Correctitud}
	% ----------- DEMO -----------
\begin{subsubsection}{Demo}

Vamos a demostrar que si tenemos una secuencia de máquinas óptima, que no esté
ordenada de mayor a menor según el tiempo de producción, podemos cambiar
el orden de dos máquinas adyacentes que estén desordenadas y seguir teniendo
una solución óptima.

Una secuencia de máquinas óptima es una secuencia tal que el Tiempo
de producción de la misma es menor o igual que el de cualquiera de sus
permutaciones.

Definimos una función T para obtener el tiempo total de procesamiento de
una máquina en la posición k de la secuencia S como

\begin{center}
$T(S,k)$ = $\left( \displaystyle\sum\limits_{i=0}^k Costo(S_i) \right)$  + $Peso(S_k)$\\
\end{center}


Definimos la función para calcular el tiempo total de procesamiento de
una secuencia de máquinas como
\begin{center}
$Tiempo(S)$ = max \{ $T(S,j)$, $j$ $\in$ [0 , $\vert S \vert$ )\}
\end{center}


Supongamos que tenemos una secuencia $S$ óptima que no se encuentra ordenada.
Entonces existe un $j$ $<$ $\vert S - 1\vert$ / $Peso(S_j)$ $<$ $Peso(S_{j+1})$.\\


Sea $S'$ la secuencia que se forma de intercambiar la máquina de la posición $j$ con la de la posición $j + 1$, quiero ver que $Tiempo(S)$ $\geq$ $Tiempo(S')$.\\

Sea $i \neq j \wedge i \neq j + 1$, es trivial ver que $T(S,i)$ = $T(S',i)$. Es decir, que si el máximo no era ni $j$, ni $j + 1$, el tiempo total se mantiene igual.\\

Supongamos entonces que $Tiempo(S)$ depende de $T(S, j)$ y $T(S, j + 1)$:\\

\begin{center}

$Tiempo(S)$ = max \{ $T(S,j)$ , $T(S,j+1)$ \}

$T(S,j)$ = $\left(\displaystyle\sum\limits_{i=0}^{j-1} Costo(S_i) \right)$ + $Peso(S_j)$ + $Costo(S_j)$

$T(S,j+1) = \left(\displaystyle\sum\limits_{i=0}^{j-1} Costo(S_i) \right) + Peso(S_{j+1}) + Costo(S_{j+1}) + Costo(S_j)$

Por hipótesis, vale que $Peso(S_j) < Peso(S_{j+1})$

$\Downarrow$


$Tiempo(S)$ = $T(S,j+1)$
\end{center}



\begin{center}
$Tiempo(S')$ = max \{ $T(S',j)$ , $T(S',j+1)$ \}


Como en $S'$ intercambiamos la máquina en la posición $j$ por la máquina en la posición $j + 1$, vale que $S'_j = S_{j+1} \wedge S'_{j + 1} = S_j$

Luego, reemplazando en T las máquinas

$T(S',j)$ = $\left(\displaystyle\sum\limits_{i=0}^{j-1} Costo(S'_i) \right)$ + $Peso(S_{j+1})$ + $Costo(S_{j+1})$

$T(S',j+1)$ = $\left(\displaystyle\sum\limits_{i=0}^{j-1} Costo(S'_i) \right)$ + $Peso(S_j)$ + $Costo(S_j)$ + $Costo(S_{j+1})$

\end{center}

\begin{center}

$Peso(S_{j+1})$ + $Costo(S_{j+1})$ + $Costo(S_j)$ $>$ $Peso(S_j)$ + $Costo(S_j)$ + $Costo(S_{j+1})$\\
$\Downarrow $

$Tiempo(S)$ $\geq$ $Tiempo(S')$\\

\end{center}

\end{subsubsection}


    \pagebreak


% ------------ RESULTADOS ------------
    \subsection{Resultados}

    \pagebreak


% ------------ CONCLUSIONES ------------
    \section{Conclusiones}
%	\input{conclusiones}
    \pagebreak
    
% ------------ CONCLUSIONES ------------
    \section{Librerías básicas}
%	\input{conclusiones}
    \pagebreak

% ------------ REFERENCIAS ------------
    \section{Referencias}
%	\input{referencias}
    \pagebreak

\end{document}
